\documentclass[10pt,a4paper]{article}
\usepackage[top=2cm,bottom=2cm,right=3cm,left=3cm]{geometry}
\usepackage[utf8]{inputenc}
\usepackage{amsmath}
\usepackage{amsfonts}
\usepackage{amssymb}
\usepackage{verbatim}
\usepackage{tikz}
\renewcommand{\familydefault}{\sfdefault}

\usepackage{listings,color}

\definecolor{verbgray}{gray}{0.9}

\lstnewenvironment{code}{%
  \lstset{backgroundcolor=\color{verbgray},
  frame=single,
  framerule=0pt,
  basicstyle=\ttfamily,
  columns=fullflexible}}{}

\definecolor{shadecolor}{rgb}{.9, .9, .9}

\begin{document}
\renewcommand{\familydefault}{\sfdefault}

General workflow for running \textit{ab initio} chemical shift calculations
with \texttt{karminus}:
\begin{enumerate}
\item Decide on a set of reference compounds (from here on termed the training set) and add all geometries 
in the xyz/ directory. Delete all prior .xyz files in that directory.
\item Add the experimental chemical shifts for the compounds to\newline
\verb!experiment/exp_chemical_shifts.py! according
to the scheme described in the comments of that file.
\item Execute
\begin{code}
python actions/calibrate_chemical_shifts.py --method <method> \
  --basis_set <basis_set>
\end{code}
which will perform all calculations necessary and will automatically save the generated output in \verb!output\!.
\item Execute
\begin{code}
python actions/report_calibration.py --method <method> \
  --basis_set <basis_set> --nuclei_type <nucleus>
\end{code}
which will extract the chemical shifts, average homotopic protons, perform a calibration and finally report the calibration and core statistics about the success of the chemical shift calibration.
A plot is also generated, that can be saved in several graphics formats.\\
The calibration is now done, and can be applied to new compounds via the formula
	$\delta (\sigma) = \sigma \cdot \text{slope} + \text{intercept}$.
Alternatively, the following steps can be performed to automate the process of nmr calculation and calibration:
\item Run the chemical shift calculation jobs on which the
calibration should be applied via
\begin{code}
actions/run_job.py --method <method> --basis_set <basis_set> \
  --xyz_file <absolute-path-to-xyzfile.xyz>
\end{code}
which will generate output in \verb!output/!.
\item Finally, the calibration can then be applied to the calculated chemical shifts:
\begin{code}
actions/apply_calibration.py --method <method> \
  --basis_set <basis_set> --job_name <jobname> \
  --nuclei_type <nucleus>
\end{code}
Which will print out the calibrated chemical shifts.
\end{enumerate}


\end{document}



